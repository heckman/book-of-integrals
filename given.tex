{
\def\no{I}
\def\theintegral{These integrals can be used without explanation}

\NewDocumentCommand{\givenintegral}{
s%         star means no newline
m%         number
O{\int}%   integral sign
m%         integrand
O{\dif u}% differential sign
m%         solution
O{\enspace+\enspace{C}}% suffix
}
{
  % \text{\raisebox{-2em}{\rule{0pt}{4em}}
\text{\color{gray}{\scriptsize{#2}:}}
&&
{\color{gray}{#3}}
\displaystyle%
\quad {#4} \quad {\color{gray}{#5}}
\quad &= \quad
{#6}{\color{gray}{#7}}
\IfBooleanF{#1}{\\}
}%
% \setlength{\jot}{0.618em}
\setlength{\jot}{0.45em}
\begin{align*}
\givenintegral{1}{u}[\dif{v}]{uv-\int{v}\dif{u}}[]
\givenintegral{2}{ u^n }{ \frac{u^{n+1}}{{n+1}}}[\enspace+\enspace{C},\enspace n\ne-1]
\givenintegral{3}   { \frac{1}{u} }   {  \ln\abs{u} }
\givenintegral{4}   { e^u }           {  e^u }
\givenintegral{5}   { a^u }           {  \frac{ a^u }{ \ln a } }
\givenintegral{6}   { \sin{u} }       { -\cos{u} }
\givenintegral{7}   { \cos{u} }       {  \sin{u} }
\givenintegral{8}   { \sec^2u }       {  \tan{u} }
\givenintegral{9}   { \csc^2u }       { -\cot{u} }
\givenintegral{10}  { \sec{u}\tan{u} }{  \sec{u} }
\givenintegral{11}  { \csc{u}\cot{u} }{ -\csc{u} }
\givenintegral{103} { \sinh{u} }      {  \cosh{u} }
\givenintegral*{104}{ \cosh{u} }      {  \sinh{u} }
\end{align*}
}
